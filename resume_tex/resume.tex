%-------------------------
% Resume in Latex
% Author : Sourabh Bajaj + some brand new features from Mary Feofanova
% Website: https://github.com/sb2nov/resume
% License : MIT
%------------------------
\PassOptionsToPackage{table}{xcolor}
\documentclass[letterpaper,10pt]{article}

\usepackage[russian]{babel}  % Включаем пакет для поддержки русского языка  

\usepackage{makecell}
\usepackage[link=off]{phonenumbers}

\usepackage{latexsym}
\usepackage[empty]{fullpage}
\usepackage{titlesec}
\usepackage{marvosym}
\usepackage[usenames,dvipsnames]{color}
\usepackage{verbatim}
\usepackage{enumitem}
\usepackage[pdftex]{hyperref}
\usepackage{fancyhdr}


\pagestyle{fancy}
\fancyhf{} % clear all header and footer fields
\fancyfoot{}
\renewcommand{\headrulewidth}{0pt}
\renewcommand{\footrulewidth}{0pt}
\usepackage[margin=0.3in]{geometry}
% Adjust margins
\addtolength{\oddsidemargin}{-0.0in}
\addtolength{\evensidemargin}{-0.0in}
\addtolength{\textwidth}{0in}
\addtolength{\topmargin}{20pt}
\addtolength{\textheight}{0.0in}

\urlstyle{same}

\usepackage{xcolor}% http://ctan.org/pkg/xcolor
\usepackage{hyperref}% http://ctan.org/pkg/hyperref
\hypersetup{
  colorlinks=true,
  linkcolor=blue!50!red,
  linkbordercolor=red,
  urlcolor=blue!70!black
}

\raggedbottom
\raggedright
\setlength{\tabcolsep}{0in}

% Sections formatting
\titleformat{\section}{
  \vspace{-10pt}\scshape\raggedright\large
}{}{0em}{}[\color{black}\titlerule \vspace{-7pt}]

%-------------------------
% Custom commands
\def \ifempty#1{\def\temp{#1} \ifx\temp\empty }

\newcommand{\resumeItem}[2]{
  \item\small{
  	\ifempty{#1}#2\else\textbf{#1}{: #2 \vspace{-2pt}}\fi
  }
}

\usepackage[dvipsnames]{xcolor}
\definecolor{mygray}{gray}{0}
\usepackage{fancybox}

\usepackage{lmodern}
\usepackage{tikz}

% Style definition
\tikzset{rndblock/.style={rounded corners,rectangle,draw,outer sep=0pt}}

% Command Definition
% 1 optional to customize the aspect, 2 mandatory: text to be framed
\newcommand{\tframed}[2][]{\tikz[baseline=(h.base)]\node[rndblock,#1] (h) {\color{black}{#2}};}

\newcommand*{\mystrut}{\rule[-0.2\baselineskip]{0pt}{0.8\baselineskip}}
\newcommand{\skill}[1]{\tframed[lightgray]{\mystrut#1}}


\newcommand{\resumeSubheading}[4]{
  \vspace{-1pt}\item
    \begin{tabular*}{0.97\textwidth}{l@{\extracolsep{\fill}}r}
      \textbf{#1} & \textcolor{mygray}{#2} \\
      \textit{\small#3} & \textcolor{mygray}{\textit{\small #4}} \\
    \end{tabular*}\vspace{-5pt}
}

\newcommand{\resumeExpSubheading}[5]{
  \vspace{-1pt}\item
    \begin{tabular*}{0.97\textwidth}{l@{\extracolsep{\fill}}r}
      \textbf{#1}  & \textcolor{mygray}{#2} \\
      \textit{\small#3} & \textcolor{mygray}{\textit{\small #4}} \\
      {\scriptsize#5}
    \end{tabular*}\vspace{4pt}
}

\newcommand{\resumeProjSubheading}[4]{
  \vspace{-1pt}\item
    \begin{tabular*}{0.97\textwidth}{l@{\extracolsep{\fill}}r}
      \textbf{#1}  & \textcolor{mygray}{#2} \\
      \scriptsize {#3} & \textcolor{mygray}{\textit{\small #4}} \\
    \end{tabular*}\vspace{4pt}
}

\newcommand{\resumeSubItem}[2]{\resumeItem{#1}{#2}\vspace{-4pt}}

\renewcommand{\labelitemii}{$\circ$}

\newcommand{\resumeSubHeadingListStart}{\begin{itemize}[leftmargin=*]}
\newcommand{\resumeSubHeadingListEnd}{\end{itemize}}
\newcommand{\resumeItemListStart}{\begin{itemize}[leftmargin=0.2in]}
\newcommand{\resumeItemListEnd}{\end{itemize}\vspace{-5pt}}

\usepackage{changepage}
\newcommand{\resumeDesc}[1]{\begin{adjustwidth}{5pt}{0pt}\vspace{-2pt}{\small{#1}}\end{adjustwidth}}

%-------------------------------------------
%%%%%%  CV STARTS HERE  %%%%%%%%%%%%%%%%%%%%%%%%%%%%


\begin{document}

%----------HEADING-----------------
\begin{tabular*}{\textwidth}{l@{\extracolsep{\fill}}r}
  \textbf{\Large Павел Ковалёв} & Email : \href{mailto:Pavel-K16@yandex.ru}{Pavel-K16@yandex.ru}\\
  Github: \href{https://github.com/PavelKovalev1444}{PavelKovalev1444} & Телефон : +7\hspace{0.5ex}981\hspace{0.5ex}191\hspace{0.5ex}85\hspace{0.5ex}73 \\
\end{tabular*}


%----------HEADING-----------------
\section{Личная информация}
  \resumeSubHeadingListStart
    \resumeSubheading
      {Гражданство:}{Республика Беларусь}
      {Временная регистрация:}{Санкт-Петербург, Российская Федерация}
    \resumeSubheading
      {Ожидаемая зарплатная вилка:}{30-50 тыс. рублей}
      {Ожидаемая должность:}{Full-stack разработчик, Frontend-разработчик}
    \resumeSubheading
      {Дата рождения:}{12.07.2002}
      {Возраст:}{20 лет}
    \resumeProjSubheading
      {О себе:}{}
      {\skill{Исполнительный} \skill{Ответственный} \skill{Мотивированный} }{}
          \resumeDesc{На данный момент заканчиваю обучение в ВУЗе СПбГЭТУ `ЛЭТИ'' по направлению обучения `Программная инженерия''. Хочу развиваться в области Frontend разработки, активно ищу работу.}
  \resumeSubHeadingListEnd


%-----------EDUCATION-----------------
\section{Образование}
  \resumeSubHeadingListStart
    \resumeSubheading
       {Санкт-Петербургский государственный электротехнический университет}{Санкт-Петербург, Россия}
      {Бакалавриат}{Сентябрь 2019 - Август 2023}
  \resumeSubHeadingListEnd


%-----------EXPERIENCE-----------------
\section{Опыт}
  \resumeSubHeadingListStart
      \resumeExpSubheading
      {Dell-EMC}{Санкт-Петербург, Россия (удалённо)}
      {Программист-разработчик}{Январь 2022 - Август 2022}
      {\skill{C} \skill{HLASM} \skill{z/OS} \skill{REXX} \skill{JCL} \skill{C++}}
      \resumeDesc{
      \begin{itemize}
          \item Разработка анализатора кода, независимого от операционной системы z/OS. Обработка GTF-трассы на языке С, настройка JCL-заданий, постановка SLIP-ловушки в системе для получения трассы.
      \end{itemize}}    
  \resumeSubHeadingListEnd

\section{Проекты}
  \resumeSubHeadingListStart
  
      \resumeProjSubheading
      {\href{https://github.com/PavelKovalev1444/estate}{Веб-приложение ``Каталог жилого фонда СПб``}}{}
      {\skill{React} \skill{JavaScript} \skill{Node.js} \skill{HTML} \skill{CSS}}{Ноябрь. 2022 - Декабрь 2022}
          \resumeDesc{Веб-приложение, представляющее из себя каталог жилого фонда СПб. Приложение позволяет выгрузить каталог жилого фонда в виде .csv файла, а также загрузить файл с информацией об объектах жилого фонда. Приложение позволяет просмотреть список объектов в виде таблицы, посмотреть их местоположение на карте, а также отображает статистику по жилому фонду. Клиентская часть приложения написана на React, серверная - на Node.js.}
  
    \resumeProjSubheading
      {\href{https://github.com/PavelKovalev1444/VK-test-task}{Сапёр}}{}
      {\skill{React} \skill{JavaScript} \skill{HTML} \skill{CSS}}{Март 2023 - Апрель 2023 (в процессе)}
          \resumeDesc{Игра ``Сапёр``, написанная на React.}
    
%    \resumeProjSubheading
%      {\href{https://github.com/PavelKovalev1444/music-learning-web-app}{Веб-приложение для изучения музыки}}{}
%      {\skill{React} \skill{JavaScript} \skill{HTML} \skill{Node.js} \skill{CSS}}{Март 2023 - май 2023 (в процессе)}
%          \resumeDesc{Веб-приложение для изучения музыки. Бакалаврская работа. На данный момент есть только макеты интерфейса.}
  \resumeSubHeadingListEnd

\section{Достижения}
  \resumeSubHeadingListStart
    \resumeProjSubheading
      {Сертификат по немецкому языку}{СПбГЭТУ `ЛЭТИ'', Санкт-Петербург, Россия}
      {Срок обучения на курсах немецкого языка:}{Февраль 2022 - Июнь 2022}
          \resumeDesc{В июне 2022 года были пройдены курсы немецкого языка и был получен сертификат уровня А2.}
  \resumeSubHeadingListEnd


%--------PROGRAMMING SKILLS------------
\section{Профессиональные навыки}
 \resumeSubHeadingListStart
   \item{
     \textbf{Языки}{: C, JavaScript, TypeScript, Python, SQL, HTML, CSS. }
   }\vspace{-7pt}
   \item{
     \textbf{Технологии}{: SQL, MongoDB, Python, Git, \LaTeX, Docker, React, Vue, Redux, Vuex. }
   }
   \vspace{-7pt}
   \item{
     \textbf{Общие знания}{: Алгоритмы, Структуры данных, ООП.}
   }
   \vspace{-7pt}
   \item{
     \textbf{Иностранные языки}{: Английский (B2), Немецкий (A1).}
   }
 \resumeSubHeadingListEnd


%-------------------------------------------
\end{document}
